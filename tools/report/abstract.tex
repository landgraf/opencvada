\abstract
Under development at Mälardalens Högskola is a Field-programmable Gate Array camera system for computer stereo vision. To ease the testing and development of said system, Ada bindings for the Open Source Computer Vision library was designed and implemented. These bindings together with a communication protocol for the device will allow for testing algorithms on images from the camera system. This will allow for the testing algorithms without writing the VHDL code this should speed up development and make sure algorithms are properly tested.
\\
To make this a possibility we have worked on finding the best ways of interfacing C and C++ code with Ada. Then with that knowledge we created bindings for Ada to the C version of Open Source Computer Vision library, The C version was chosen due to the limitations of interfacing C++ with Ada that are discussed in the report. Also benchmarks was created to test the performance of the Ada version compared to C and Python. From the benchmarks we found that the performance results between C and Ada is about equal, sometimes the Ada performance is better and other times the C performance is better while the Python bindings are a bit behind in all cases. 
\abstract
Under utveckling på Mälardalens Högskola är ett Field-programmable Gate Array kamera system för stereo seende för datorer. För att underlätta testningen och utvecklingen av det systemet så har Ada bindningar till Open Source Computer Vision biblioteket designats och implementerats Dessa bindningar tillsammans med ett kommunikations protokoll för systemet tillåter testande av algoritmer på bilder från kamera systemet. Detta kommer att tillåta testning av algoritmer utan att skriva VHDL kod, detta borde snabba upp utvecklingen och försäkra om att algoritmerna är ordentligt testade.
\\
För att kunna göra detta möjligt har vi arbetat med att finna det bästa sättet att skapa detta gränssnitt mellan C eller C++ och Ada. Med den kunskapen så har vi skapat bindningar mellan Ada och C versionen av Open Source Computer Vision biblioteket. C versionen valdes på grund av  begränsningarna med att skapa ett gränssnitt mellan C++ och Ada. Vi skapande även ett tester för att testa prestandan om man jämför Ada versionen med C eller Python versionerna. Från testerna så kan vi se att skillnaden i prestanda mellan Ada och C är försumbar, Ada har bättre prestanda ibland och C vid andra tester, medans Python versionen har lägst prestanda i alla tester.