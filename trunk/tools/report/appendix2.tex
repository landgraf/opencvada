\chapter{Demo Application}
\section{Demo.gpr}
\begin{lstlisting}
with "OpenCvAda";
	-- A demo application to show how to use OpenCvAda
project Demo is
	for Source_Dirs use ("Source\Demo");
	for Object_Dir use "Build";
	for Exec_Dir use "Exec";
	for Languages use ("Ada");

	package Builder is
		for Default_Switches 
			("ada") use ("-s", "-m");
	end Builder;

	package Compiler is
		for Default_Switches ("ada") 
			use ("-gnatw.hkd", "-gnat05");
	end Compiler;

	for Main use ("demo.adb");
end Demo;
\end{lstlisting}
\section{Demo.adb}
\begin{lstlisting}
with Highgui; 			use Highgui;
with Core; 			use Core;
with Core.Operations; 		use Core.Operations;
with Imgproc; 			use Imgproc;
with Imgproc.Operations; 	use Imgproc.Operations;

procedure Demo is
	-- Parsed variables created from the trackbars
	Aperture_Size		: Integer := 1;
	Block_Size 		: Integer := 1;

	-- Unparsed variables used with the trackbars
	Slider_A_Value 		: aliased Integer := 1;
	Slider_B_Value 		: aliased Integer := 1;   

	-- Variables used with the camera.
	Capture 		: aliased Cv_Capture_Ptr;
	Image 			: Ipl_Image_Ptr;

	-- Temporary images
	Corners, Image_Gray 	: aliased Ipl_Image_Ptr;
   	
	-- Used to convert a 32F image to a 8U image.
	Scale, Shift		: Long_Float;
	Min, Max 		: aliased Long_Float;

	-- Change the convention to match a C function.
	procedure Slider_A (Position : Integer);
	pragma Convention (C, Slider_A);   

	procedure Slider_A (Position : Integer) is
	begin
		-- Value is 0 or even.
		if Position = 0 or Position mod 2 = 0 then 
			Aperture_Size := Position + 1;
		else
			Aperture_Size := Position;
		end if;
	end Slider_A;

	-- Change the convention to match a C function.
	procedure Slider_B (Position : Integer);
	pragma Convention (C, Slider_B);

	procedure Slider_B (Position : Integer) is
	begin
		if Position > 0 then
			Block_Size := Position;
		else
			Block_Size := 1;
		end if;
	end Slider_B;

begin
	-- Create Windows, 
	-- Corners window will be resizeable. 
	Cv_Named_Window ("Original",
			 Highgui.Cv_Window_Autosize);
	Cv_Named_Window ("Corners", 
			 Highgui.Cv_Window_Normal);

	-- Move the windows, 
	-- so they are not on top of eachother.
	Cv_Move_Window ("Corners", 700, 50);
	Cv_Move_Window ("Original", 0, 50);

	Cv_Create_Trackbar("Aperture",
			   "Corners",
			   Slider_A_Value'Access, 
			   31, -- Max value.
			   Slider_A'Unrestricted_Access);
  	Cv_Create_Trackbar("Block", 
			   "Corners", 
			   Slider_B_Value'Access,
			   11, -- Max value.
			   Slider_B'Unrestricted_Access);
	Capture := Cv_Create_Camera_Capture (0);

	loop
		exit when Cv_Wait_Key (100) = Ascii.Esc;
		Image := Cv_Query_Frame (Capture);
		Cv_Show_Image ("Original", Image);
		Corners := Cv_Create_Image
			(Cv_Get_Size(Image),
			Ipl_Depth_32f, 
			1);
		Image_Gray := Cv_Create_Image
			(Cv_Get_Size (Image),
			Ipl_Depth_8u,1);
		Cv_Cvt_Color
			(Image, Image_Gray, Cv_Rgb2gray);
		Cv_Corner_Harris(Image_Gray, 
				 Corners, 
				 Block_Size, 
			 	 Aperture_Size);
		Cv_Min_Max_Loc (Corners, 
			   	Min'Access, 
				Max'Access, 
				null, null, null);

		Scale := 255.0 / (Max - Min);
		Shift := -Min * Scale;

		Cv_Convert_Scale 
			(Corners, 
			Image_Gray, Scale, Shift);
		Cv_Show_Image ("Corners", Image_Gray);
		Cv_Release_Image (Image_Gray'Access);
		Cv_Release_Image (Corners'Access);
	end loop;
	Cv_Destroy_Window ("Original");
	Cv_Destroy_Window ("Corners");
	Cv_Release_Capture (Capture'Access);
end Demo;
\end{lstlisting}